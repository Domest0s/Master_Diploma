\newpage

\section*{Анотація}

  {\bf Яригіна О. А.} "\thesistopicUA"\\
  {\itshape Кваліфікаційна робота магістра за спеціальнітю 6.070101 --- фізика, спеціалізація "квантова теорія поля". --- Київський національний університет імені Тараса Шевченка, фізичний факультет, кафедра квантової теорії поля. --- Київ, 2015.\\}
  {\itshape \bfseries Науковий керівник:} д. ф.-м. н., проф. Двалі Г. \\%[0.5cm]

  У роботі розглянуті деякі наслідки застосування корпускулярного формалізму до розгляду чорних дір. Як один з прикладів досліджено розсіяння зовнішнього скалярного поля на чорній дірі з ненульовим баріонним зарядом. Також у даному формалізмі розглянута задача народження частинок при взаємодії зі швидко осцилюючим полем інфлатона під час постінфляційного розігріву Всесвіту з врахуванням зворотньої реакції народжених частинок на систему. Щоб проілюструвати кардинальну різницю структури чорних дір від систем таких як планети або зірки, розглянута задача збудження гравітонів у гравітаційно взаємодіючих фізичних системах.
   \\[0.5cm]
  {\bf Ключові слова:} чорна діра, гравітон, когерентні стани, скірміон, розсіяння, постінфляційний розігрів.\\

\section*{Аннотация}

  {\bf Ярыгина О. А.} "\thesistopicRU"\\
  {\itshape Квалификационная работа магистра по специальности 6.070101 --- физика, специализация "квантовая теория поля". --- Киевский национальний университет имени Тараса Шевченка, физический факультет, кафедра квантовой теории поля. --- Киев, 2015.\\}
  {\itshape \bfseries Научный руководитель:}  д. ф.-м. н., проф. Двали Г.\\%

  В работе рассмотрены некоторые последствия применения корпускулярного формализма к чёрным дырам. Как один из примеров, рассмотрено рассеяние внешнего скалярного поля на чёрной дыре з ненулевым барионным зарядом. Также в данном формализме рассмотрена задача рождения частиц при взаимодействии с быстро осциллирующим полем инфлатона в период постинфляционного разогрева Вселенной с учтетом обратной реакции рожденных частиц на систему. Для иллюстрации кардинальной разницы структуры чёрных дыр от систем таких как планеты и звёзды, рассмотрена задача возбуждения гравитонов в гравитационно взаимодействующих физических системах.
   \\[0.5cm]
  {\bf Ключевые слова:} чёрная дыра, гравитон, когерентные состояния, скирмион, рассеяние, постинфляционный разогрев.\\[0.5cm]

\section*{Summary}

  {\bf Iarygina O. A.} "\thesistopicEN"\\
  {\itshape Qualifying work of the magister on a speciality 6.070101 --- physics, specialization "quantum field theory". --- National Taras Shevchenko University of Kyiv, Faculty of Physics, Department of Quantum Field Theory. --- Kyiv, 2015.\\}
  {\itshape \bfseries Research supervisor:}  Prof. Dr. Georgi Dvali \\%[0.5cm]

In the project we considered some of consequences of black hole description in the corpuscular formalism. As one of examples, we considered external scalar field scattering on a black hole with a non-zero barion number. In addition, we discussed the particle creation problem caused by interaction with fast-oscillating inflaton field during the cosmological reheating, taking into account the back-reaction of created particles onto the system. For illustration a significant difference in the structure of black holes and systems such as planets or stars, we considered the problem of exiting single gravitons in gravitationally interacting physical systems. \\[0.5cm]
 {\bf Key words:} black hole, graviton, coherent states, skirmion, scattering, cosmological reheating.\\
