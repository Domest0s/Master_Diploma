\newpage

\section*{Анотація}

  {\bf Березюк І.Б.} "\thesistopicUA"\\
  {\itshape Кваліфікаційна робота магістра за спеціальнітю 6.04020304 --- фізика, спеціалізація "Фізика ядра та фізика високих енергій". --- Київський національний університет імені Тараса Шевченка, фізичний факультет, кафедра ядерної фізики. --- Київ, 2015.\\}
  {\itshape \bfseries Науковий керівник:} д. ф.-м. н., Massimiliano  Ferro-Luzzi \\%[0.5cm]

	У роботі розглянуто дизайн дрейфових трубок, що допускають значне провисання дроту в трубці. В програмному пакеті Garfield для такої моделі трубок була здійснена оцінка профілю провисання дроту під дією гравітаційного і електричного полів та розроблена методика для визначення профілю трубки експериментально на основі аналізу DT-розподілів. Була проведена оцінка зміни точності реконструкцій горизонтальних треків мюонів енергії 1 ГеВ  для для випадків центрованого та зміщеного дроту - точність реконструкції перпендикулярних до напряму провисання треків не змінилася. Також було проведено виміри коефіцієнту підсилення в прототипі трубки. Виміри показали, що значення газового коефіцієнту підсилення трубки на практиці в $\sim 2.5$ рази більше за очікуване з симуляцій в GARFIELD. Спектрометричні вимірювання дрейфовою трубкою $\gamma$-квантів від джерела $Fe^{55}$ показали, що ефект просторового заряду для даного джерела в трубці є значним.
   \\[0.5cm]
  {\bf Ключові слова:} дрейфова трубка, аргон, провисання, дріт, мюон, час дрейфу, TR-розподіл, просторовий заряд.\\

\section*{Аннотация}

  {\bf Березюк И.Б.} "\thesistopicRU"\\
  {\itshape Квалификационная работа магистра по специальности 6.04020304 --- физика, специализация "физика ядра и физика высоких энергий". --- Киевский национальний университет имени Тараса Шевченка, физический факультет, кафедра ядерной физики. --- Киев, 2015.\\}
  {\itshape \bfseries Научный руководитель:}  д. ф.-м. н. Massimiliano  Ferro-Luzzi\\%

	В работе рассмотрены дизайн дрейфовых трубок, допускающих значительное провисание проволки в трубке. В программном пакете Garfield для такой модели трубок была осуществлена ​​оценка профиля провисания провода под действием гравитационного и электрического полей и разработана методика для определения профиля трубки экспериментально на основе анализа DT-распределений. Была проведена оценка изменения точности реконструкции горизонтальных треков мюонов энергии 1 ГэВ для для случаев центрированного и смещенного положения проволоки - точность реконструкции треков  перпендикулярных к направлению провисания не изменилась. Также было проведено измерения коэффициента усиления в прототипе трубки. Измерения показали, что значение газового коэффициента усиления трубки на практике в $ \sim 2.5$ раза больше ожидаемого из симуляций в GARFIELD. Спектрометрические измерения дрейфовой трубкой $\gamma$-квантов от источника $Fe^{55}$ показали, что эффект пространственного заряда для этого источника в трубке значителен.
   \\[0.5cm]
  {\bf Ключевые слова:} дрейфовая трубка, аргон, провисание, проволка, мюон, время дрейфа, TR-распределение, пространственный заряд.\\[0.5cm]

\section*{Summary}

  {\bf Bereziuk I.B.} "\thesistopicEN"\\
  {\itshape Qualifying work of the magister on a speciality 6.04020304 --- physics, specialization "Nuclear physics and high energy physics". --- National Taras Shevchenko University of Kyiv, Faculty of Physics, Department of Nuclear physics. --- Kyiv, 2015.\\}
  {\itshape \bfseries Research supervisor:}  Prof. Dr. Georgi Dvali \\%[0.5cm] 
 The design of the straw tube allowing considerable wire sagging in the tube is discussed in this paper. Calculation of wire sagging profile under gravitational and electric field was performed in GARFIELD program for such kind of tube design.
Developed a method for sag profile determining based on analysis of drift time distributions from the tube. Also was performed an estimation of detector accuracy  for sag-allowing design method and comparison with accuracy from "wire in the center" design. 
	There was also a measure of the gain in the prototype tubes. Measurements showed that the value of the gas gain with factor of $\sim 2.5$ higher than expected from the GARFIELD simulations. Spectrometric measurements of the $\gamma$ rays from the radioactive source $Fe^{55}$ showed that the effect of the space charge for this source is significant.\\[0.5cm]
 {\bf Key words:} drift tube, argon, sagging, wire, muon, drift time, TR-relation, space sharge.\\

